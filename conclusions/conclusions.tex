\section{Discussion and conclusions}
\protect\label{section:discussion}

A survey of the data used in other papers confirms the rotation period of $2.87
\pm 0.01$ days derived in most of the papers reviewed and this period is
also confirmed by the processing of the {\rem} data.

The lack of contrast in the {\rem} images is unfortunate, but could be improved
by increasing the exposure time, which would improve the SNR considerably,
especially for the fainter reference stars.

Nevertheless, it is clear that it is possible to recover the same period,
believed to be the rotation period, from \ross.

Further developments in the processing of {\rem} data will include consideration
of the other target \rdwarf s, {\prox} and {\bstar} in addition to the other
filters and the REMIR data. The optical filters \texttt{g}. \texttt{i} and \texttt{z} are likely to be
less promising than \texttt{r}, however as \texttt{g} is on a much noisier
portion of the CCD and the other too filters do not show nearly as many
reference stars as the \texttt{g} and \texttt{r} filters.
