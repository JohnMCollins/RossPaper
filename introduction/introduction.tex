\section{Introduction}
\protect\label{section:intro}

{\ross} is 2.976 pc distant, of spectral type M3.5. It has a reasonably high
proper motion of $-637.02$ mas/yr in Right Ascension and  {–}191.64 mas/yr
in Declination and a radial velocity of {–}10.7 km/s \citep{vanleeuwen07}. It is
notable for its strong activity, for example in \citet{wargelin08} and it is
well-understood that activity is associated with a fast rotation period,
especially in \rdwarf s, for example in \citet{mohanty03}.

As part of a project to analyse automatically-collected data from the {\rem}
telescope in La Silla, Chile, further described below in Section
\ref{section:rem}, which covers 3 \rdwarf s including \ross, existing data for
those stars were studied. It appeared that there were some omissions and
discrepancies between the various tabulations of parameters for \ross, which
this paper seeks to address.

\subsection{Data relevant to \ross}

For the purposes of discussion in this paper, the following parameters for
{\ross} are of particular interest.

\begin{enumerate}
  \item The radius.
  \item The rotational velocity \vsini.
  \item The rotation period.
\end{enumerate}

These are all inter-related, in that the rotational velocity (without the factor of
the sine of the inclination) is related to the others by $v = \frac{2 \pi
r}{t}$, where $v$ is the ``true'' rotational velocity as opposed to the
projected rotational velocity, \vsini, $r$ is the radius and $t$ is the rotation
period.

\subsubsection{Values for radius}

The radius of a star is calculated by determining the bolometric luminosity and
the temperature and applying the black-body radiation formula $L = 4 \pi R^2
\sigma T^4$, where L is the luminosity, which can be derived from the absolute
magnitude, T is the temperature in {\degree}K and $\sigma$ is the
Stephan-Boltzmann constant.

A radius was previously given in \citet{pettersen80} of $1.5 \pm 0.2 \times 10^8$ m, which equates to $0.22 \pm
0.03$\rsun. A larger radius of $0.24 \pm 0.06$ {\rsun} is given in \citet{johnson83}
\footnote{\citet{johnson83} do not explicitly state the
uncertainty, but gives a limit of 25\% for results not studied in more detail
from which this uncertainty was calculated.}.

A much more recent study, \citet{pineda21} has proposed a radius of $0.200 \pm
0.008$ \rsun. The discrepancies between these values are accounted for by
refinements to the measurements of bolometric luminosity and the effective
temperature. In the latter paper these.are given as $1.537 \pm 0.018
\times 10^31$ erg/s\footnote{This is equivalent to $0.00402 \pm 0.00005$\lsun.}
and $3248^{+68}_{-66}$K.

There does appear to be considerable variation in the values for bolometric
luminosity and/or magnitude between papers and also in the effective
temperature, which accounts for the considerable variation in values.

\subsubsection{Rotational velocity \vsini}

As with the radius, there is some variation in the rotational velocity \vsini.

A value of $3.5 \pm 0.5$ km/s is given in \citet{johnskrull96} and quoted in
\citet{wargelin08} who state that the fast rotation period is indicative of an
age of less than 1 Gyr. The period of rotation is not stated there.
\citet{reiners18} report a value for {\vsini} of $3.0 \pm 1.5$ km/s, whilst
\citet{hojjatpanah19} report a figure of $5.20 \pm 0.91$.

These figures are all obtained from spectral line broadening. It is possible
that the star's known considerable activity might contribute to this yielding an
overestimate.

\subsubsection{Rotation period}

The value of the rotational velocity sets an upper limit on the rotation period.
If the inclination of the star is less than 90{\degree}, then the projected
rotational velocity {\vsini} will be lower than the actual rotational velocity
by the factor of the sine of the inclination.

Using the formula $t = \frac{2 \pi r}{v}$ and assuming a radius $r$ of
0.200\rsun, an upper limit on the rotation period is obtained of 2.891 days if
the value of {\vsini} of 3.5 km/s is assumed, 3.373 days if 3.0 km/s is assumed
and 1.946 days if 5.2 km/s is assumed.

The literature reports varying calculations of the rotation period.
Previously, in \citet{jarrett76}, an activity cycle of about 2 days was reported, but
possibly this might have been confused with the rotation period. 
A period of 2.869 days is given in \citet{kiraga07}, one of 2.857
or 2.843 days (taken from {\ktwo} or {\MEarth} respectively) in \citet{newton18},
$4.7 \pm 2.3$ days in \citet{reiners18}\footnote{This cannot be credible as the
figure for {\vsini} also given by this paper of $3.0 \pm 1.5$ km/s sets an
upper limit of the rotation period of 3.373 days.} and $2.87 \pm 0.01$ days in
\citet{diezalonso19}, based on {\asas} data.

\subsection{Aims of this study}

In view of the plethora of statistics for \ross, it would be useful to confirm
or refine the figure for the rotation period as no planets have been reported for {\ross}
and a low inclination angle might be one of the
reasons for the difficulty in detection which would be indicated if the maximum
rotation period indicated by (\vsini) was significantly more than the detected
rotation period.

As the authors were considering the visible light observations of the {\rem}
telescope, a study of the rotation period was undertaken using the various data
sources referred to in the above papers and the results presented herein.
