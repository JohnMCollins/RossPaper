\section{Introduction}
\protect\label{section:intro}

{\ross} is 9.6 ly distant, of spectral type M3.5. It has a reasonably high
proper motion of $-637.02$ mas/yr in Right Ascension and  {–}191.64 mas/yr
in Declination and a radial velocity of {–}10.7 km/s \citep{vanleeuwen07}.

The strong activity is noted in \citet{wargelin08}, who also quote a value for
{\vsini} of of $3.5 \pm 0.5$ km/s obtained from \citet{johnskrull96} stating
that the fast rotation period is indicative of an age of less than 1 Gyr.
The period of rotation is not stated there, but if
the radius of $0.24 \pm 0.06$ solar given in
\citet{johnson83}\footnote{\citet{johnson83} do not explicitly state the
uncertainty, but gives a limit of 25\% for results not studied in more detail
from which this uncertainty was calculated.} is assumed, this would given an
upper limit of $3.5 \pm 1.0$ days, less if the angle of inclination was less than 90\degree, by a factor of the sine of the inclination.

There are some discrepancies in the reported values of \vsini, in that
\citet{reiners18} report a value of $3.0 \pm 1.5$ km/s, whilst
\citet{hojjatpanah19} report a figure of $5.20 \pm 0.91$, which would set an
upper limit on the period of $2.3 \pm 0.9$ days.

Calculations of the period range from 2.857 days or 2.843 days (taken from K2
or MEarth respectively) in \citet{newton18}, $2.87 \pm 0.01$ days in
\citet{diezalonso19} and $4.7 \pm 2.3$ days in \citet{reiners18}. Previously,
in \citet{jarrett76}, an activity cycle of about 2 days was reported, but
possibly this might refer to the rotation period.

It would be useful to confirm or refine the figure for
the rotation period as no planets have been reported for {\ross}
and a low inclination angle might be one of the
reasons for the difficulty in detection. This would be indicated if a much
shorter rotation period was detected than would be indicated by calculating the
period from $2\pi r / v$.
